\documentclass[a4paper]{article}         % 设置纸张大小
\usepackage[margin=1in]{geometry}        % 设置边距,符合Word设定
\usepackage{ctex}                        % 对中文支持的宏包
\usepackage{cite}                        % 引用
\usepackage{amsfonts}                    % 输入数集符号
\usepackage{amsmath}                     % 数学支持
\usepackage{graphicx}                    % 插入图片所需要的宏包
\usepackage{lipsum}                      % 生成虚拟文本(Lorem Ipsum)的宏包,测试或者占位用。
\usepackage{amssymb}                     % 为了得到漂亮的\geqslant \leqslant 使用的包
\usepackage{url}                         % bibtex中引用herf
\usepackage{listings}                    % 添加程序代码
\usepackage{xcolor}                      % 加以渲染代码颜色
\usepackage[colorlinks=true]{hyperref}   % [hidelinks]取消超链接颜色,并且目录有超链接。为了避免宏包冲突,将这个包放在最后调用。

\newcommand{\upcite}[1]{\textsuperscript{\cite{#1}}} % 定义上标引用,若套两层会让上标更“上”一点
\title{\heiti\zihao{2} RSA}
\author{\songti Crazy\_13754}
\date{\today}

\lstset{
	basicstyle       = \sffamily,            % 基本代码风格
	keywordstyle     = \bfseries,            % 关键字风格
	commentstyle     = \rmfamily\itshape,    % 注释的风格,斜体
	stringstyle      = \ttfamily,            % 字符串风格
	flexiblecolumns,                         % 别问为什么,加上这个
	numbers          = left,                 % 行号的位置在左边
	showspaces       = false,                % 是否显示空格,显示了有点乱,所以不显示了
	numberstyle      = \zihao{-5}\ttfamily,  % 行号的样式,小五号,tt等宽字体
	showstringspaces = false,
	captionpos       = t,                    % 这段代码的名字所呈现的位置,t指的是top上面
	frame            = lrtb,                 % 显示边框
	frame            = shadowbox,            % 阴影效果
}

\lstdefinestyle{cpp}{
	language         = C++,
	keywordstyle     = \color{blue!70},
	commentstyle     = \color{red!50!green!50!blue!50},
	rulesepcolor     = \color{red!20!green!20!blue!20},
	escapeinside     = ``,                   % 英文分号中可写入中文
	xleftmargin      = 2em,xrightmargin = 2em,aboveskip = 1em,
	framexleftmargin = 2em
}
\lstdefinestyle{Python}{
	language         = Python,               % 语言选Python
	basicstyle       = \zihao{-5}\ttfamily,
	numberstyle      = \zihao{-5}\ttfamily,
	keywordstyle     = \color{blue},
	keywordstyle     = [2]\color{teal},
	stringstyle      = \color{magenta},
	commentstyle     = \color{red}\ttfamily,
	breaklines       = true,                 % 自动换行,建议不要写太长的行
	columns          = fixed,                % 如果不加这一句,字间距就不固定,很丑,必须加
	basewidth        = 0.5em,
}

\begin{document}

% \begin{sloppypar} % Typeset the paragraphs with \sloppy in effect (see \fussy & \sloppy). Use this to locally adjust line breaking, to avoid ‘Overfull box’ or ‘Underfull box’ errors.

\maketitle
\begin{abstract}
	写了些关于rsa的东西。
\end{abstract}
\tableofcontents
\section{说在前面}

虽然本文花费大力气写了数学种种理论,笔者却深感这并无必要,因为无论作者抑或读者所关心的其实是技术,数学理论不过是些模糊空气——固然是十分重要的支撑,写起来也不容忽视,但也不妨碍人们对其漠不关心。所以,读者大可以略看繁复的证明,注意于各类技术的细节。

\section[引言]{引言}

% 实际上,写这篇的时候发现有论文写的很清楚了\upcite{chenchuanbo2006rsa}。此外,网上各种博客内容已经一应俱全,本来把它们丢上来就可以了。你问我为什么还要写这篇文章?请看这个图 \ref{fig:rsa1}:

% \begin{figure}[htbp]
% 	% [h]当前位置。将图形放置在正文文本中给出该图形环境的地方。如果本页所剩的页面不够,这一参数将不起作用。
% 	% [t]顶部。将图形放置在页面的顶部。
% 	% [b]底部。将图形放置在页面的底部。
% 	% [p]浮动页。将图形放置在一只允许有浮动对象的页面上。
% 	\centering
% 	\includegraphics[width=1.77in,height=1.75in]{contents/rsa1.jpg}
% 	\caption{This is an inserted jpg graphic}
% 	\label{fig:rsa1}
% \end{figure}

% 如果你还是没有明白我想说什么,请再看看这个表 \ref{table:1}:

% \begin{table}[h!]
% 	\centering
% 	\begin{tabular}{ c c c }
% 		\hline
% 		对写奇奇怪怪东西的看法 & 可以理解  & 不可理喻   \\
% 		\hline
% 		支持          & 0.1\% & 0.0\%  \\
% 		不支持         & 0.2\% & 99.7\% \\
% 		\hline
% 	\end{tabular}
% 	\caption{Table to test captions and labels}
% 	\label{table:1}
% \end{table}

% 如果(虽然几乎是当然的)你还是不理解,那就看看这些图\ref{jjy}\ref{TR}\ref{Lagrange}\ref{dx906999}:

% \begin{figure}[h!]
% 	\centering
% 	\begin{minipage}{0.49\linewidth}
% 		\centering
% 		\includegraphics[width=0.9\linewidth]{contents/rsa2.jpg}
% 		\caption{冬の花}
% 		\label{jjy} % 文中引用该图片代号
% 	\end{minipage}
% 	\begin{minipage}{0.49\linewidth}
% 		\centering
% 		\includegraphics[width=0.9\linewidth]{contents/rsa3.jpg}
% 		\caption{RickT}
% 		\label{TR} % 文中引用该图片代号
% 	\end{minipage}
% 	%\qquad
% 	%让图片换行,

% 	\begin{minipage}{0.49\linewidth}
% 		\centering
% 		\includegraphics[width=0.9\linewidth]{contents/rsa4.jpg}
% 		\caption{朗格拉日}
% 		\label{Lagrange} % 文中引用该图片代号
% 	\end{minipage}
% 	\begin{minipage}{0.49\linewidth}
% 		\centering
% 		\includegraphics[width=0.9\linewidth]{contents/rsa5.jpg}
% 		\caption{dx3906}
% 		\label{dx906999} % 文中引用该图片代号
% 	\end{minipage}
% \end{figure}

% \clearpage
% 这显然把事情弄得更糟。好吧,只是我在学\LaTeX,而你---我的朋友---浪费了你生命中的宝贵时间来看刚刚的内容,而且你接下来看到的东西也都可以在网上找到。不知道这篇文章所帮助你减少的在浏览器上搜索的时间能否弥补它所浪费的时间。% ! 或许,你需要的是及时止损,现在就放弃阅读?

众所周知,密码学的提出是为了保证传输信息的安全。在古代,人们使用的加密方式可能是一张字符映射表,将信息中的字符逐个替换来让信件难以解密,只要这张表格不被敌人接获,信息就将保持安全。换句话说,只要加密的算法是安全的,信息就将是安全的。而现代的加密技术则选择公开加密算法,使用密钥加密。这被称为柯克霍夫原则(Kerckhoffs's principle)。其中原因也不难想到:使用特定算法加密一旦算法泄露,加密的内容会易遇破解,而使用密钥则是“更新”了这个问题,因而可以长期反复使用。敌人如果买通工作人员,源码的泄露也难以影响其安全性。当加密用软件或硬件长期使用的时候,敌人可能通过各种方式分析其原理。公开的算法经过专家验证,人们对广泛流传的加密技术有信心。

更进一步,加密算法可以粗略分为两种:对称加密与非对称加密。对称加密技术,例如AES,使用相同密钥加密与解密,速度较快,但密钥本身通常需要非对称加密技术加密并传输。非对称加密技术,例如RSA,有公钥私钥之分,但速度较慢,一次加密的信息较少。此外,公钥密码能够解决消息鉴别问题,就是指用来检验消息来自于声称的来源并且没有被修改。

公钥体制的基础是陷门(单向函数),即某种实际处理过程的不可逆性。目前的公钥思想基于两种:一是依赖于大数的因数分解的困难性;二是依赖于求模p离散对数的困难性。RSA密码算法就是基于大数的因数分解的困难性\upcite{chenchuanbo2006rsa}。

RSA加密算法是一种非对称加密算法,在公开密钥加密和电子商业中被广泛使用。RSA是由罗纳德·李维斯特(Ron Rivest)、阿迪·萨莫尔(Adi Shamir)和伦纳德·阿德曼(Leonard Adleman)在1977年一起提出的。当时他们三人都在麻省理工学院工作。RSA 就是他们三人姓氏开头字母拼在一起组成的。1973年,在英国政府通讯总部工作的数学家克利福德·柯克斯(Clifford Cocks)在一个内部文件中提出了一个与之等效的算法,但该算法被列入机密,直到1997年才得到公开。

\href{https://mathmu.github.io/MTCAS/doc/IntegerFactorization.html}{这篇文章}介绍了因子分解的相关算法,然而,这其中的很多方法对RSA来说并不适用。对于RSA的“暴力”攻击来说,\href{https://zh.wikipedia.org/wiki/%E6%99%AE%E9%80%9A%E6%95%B0%E5%9F%9F%E7%AD%9B%E9%80%89%E6%B3%95}{普通数域筛法(GNFS)}应该是最优的。如果我能看懂这个算法的话,会加上对它的介绍的…… % TODO 这个记得要写

\section{RSA加密与解密过程} \label{RSA加密与解密过程}

小发(消息发送者)想要给小收(消息接受者)发一条需要保密的信息。

\begin{itemize}
	\item 小收取两个非常大的素数$p$和$q$,并令$N = p \cdot q$,$\varphi(N) = (p - 1) \cdot (q - 1)$。
	\item 找到一个素数$e < \varphi(N)$,且要求$e$与$\varphi(N)$互素,即有$gcd(e, \phantom{;} \varphi(N)) = 1$。 % 先找d会怎么样,e不是素数会怎么样
	\item 计算$e$在模$\varphi(N)$上的逆元$d$,即求$d$满足$e \cdot d \bmod \phantom{;} \varphi(N) = 1$。
	\item 小收将$(N, \phantom{;} e)$作为公钥 (pk, Public key) 发给对方,$(N, \phantom{;} d)$作为私钥(pk, Private key)保存。
	\item 小发接受公钥后,将原文(pt, Plaintext)通过预先设定好的方式转换成数字,记为$pt$,则密文(ct, Ciphertext)满足$ct = p ^ e \bmod \phantom{;} N$(也就是$ct \equiv pt ^ e (\bmod \phantom{;} N)$),对方将密文发回。
	\item 小收接收密文,并使用私钥解密:$pt = ct ^ d \bmod \phantom{;} N$,也就是$pt \equiv ct ^ d (\bmod \phantom{;} N)$。
	      % TODO 需要一个很好的流程图
\end{itemize}

\section{RSA数学基础与证明} \label{RSA数学基础与证明}

在无特殊说明的情况下,本章中所有的字母均指代正整数,$p, \phantom{;}p_1, \phantom{;}\cdots, \phantom{;}p_r$为素数(注意 1 不是素数)。

\subsection{同余}

同余指的是,对于$a, \phantom{;} b \in \mathbb{Z}, \phantom{;} n \in \mathbb{Z\textsuperscript{*}}$,若$\exists k \in \mathbb{Z}$,满足$a - b = k \cdot n$,则称 $a, \phantom{;} b$ 模 $n$ 同余,记作$a \equiv b (\bmod \phantom{;} n)$。你想问我为什么要用乘法定义?噢据王鲲鹏老师说,这样定义有助于我们“操作”。

\subsection{欧几里得算法}

欧几里得算法(或辗转相除法)是得到两数最大公因数的算法,其代码如下:

\lstinputlisting[
	style   = cpp,
	caption = {gcd},
	label   = {gcd.cpp}
]{contents/gcd.cpp}

下证明算法正确性。设$a > b, \phantom{;} c_1 = \gcd(a, \phantom{;} b), \phantom{;} c_2 = \gcd(b, \phantom{;} a \bmod \phantom{;} b)$。此时,$c_2 \mid b$ 且 $c_2 \mid (a \bmod \phantom{;} b)$,又因为$a = k_1b + (a \bmod \phantom{;} b)$,则$c_2 \mid a$,因此 $c_2$ 为 $a, \phantom{;} b$的因数,$c_2$ 因而为 $c_1$的因数。同理,$c_1$ 为 $c_2$的因数,即有$c_1 = c_2$。不断进行此操作,即可得到两数的最大公因数。

若$a < b$,则 $\gcd(a, \phantom{;} b) = \gcd(b, \phantom{;} a)$,同上。

\subsection{裴蜀定理} \label{裴蜀定理}
% ! 谁来教教我啊

裴蜀定理可表述为:若 $a, \phantom{;} b$ 不全为零,则 $\forall x, \phantom{;} y \in \mathbb{Z}$ , 有 $\gcd(a, \phantom{;} b) \vert ax + by$ ,且 $\exists x, \phantom{;} y$,使 $ax + by = gcd(a, \phantom{;} b)$。

因为$\gcd(a, \phantom{;} b) \vert a, \phantom{;} b$ ,所以$\forall x, \phantom{;} y \in \mathbb{Z}$, $\gcd(a, \phantom{;} b) \vert ax + by$ 一定成立。

因为 $\frac{a}{\gcd(a, \phantom{;} b)}$ 与 $\frac{b}{\gcd(a, \phantom{;} b)}$ 均为整数且互素,因而 $ax + by = gcd(a, \phantom{;} b)$ 有解可转为 $\frac{a}{\gcd(a, \phantom{;} b)} x + \frac{b}{\gcd(a, \phantom{;} b)} y = 1$,即证明 $a_1, \phantom{;} b_1$ 互素时,$a_1 x + b_1 y = 1$ 恒有解。

展开欧几里得算法可得:

\begin{gather*}
	a_1 = q_1 b_1 + r_1 \\
	b_1 = q_2 r_1 + r_2 \\
	\cdots              \\
	r_{n-2} = q_{n} r_{n-1} + r_n\\
	r_{n-1} = q_{n+1} r_{n}\\
\end{gather*}

且 $r_n = \gcd(a_1, \phantom{;} b_1) = 1$,则有:

\begin{equation}
	\label{裴蜀定理消余数方程组}
	\begin{aligned}
		1 & = r_n                                      \\
		  & = r_{n-2} - q_n r_{n-1}                    \\
		  & = r_{n-2} - q_n (r_{n-3} - q_{n-1}r_{n-2}) \\
		  & \cdots
	\end{aligned}
\end{equation}

用同样方法不断消去 $r_{n-2}, \phantom{;} r_{n-3}, \phantom{;} \cdots, \phantom{;} r_1$ 可得 $a_1 x + b_1 y = 1$ 有解。

\subsection{拓展欧几里德算法}

% https://zhuanlan.zhihu.com/p/100567253

拓展欧几里德算法就是求 $ax + by = c$ 的一组解的方法。由 \ref{裴蜀定理} 节,$c$ 一定是 $\gcd(a, \phantom{;} b)$ 的倍数。此时,重复方程 \ref{裴蜀定理消余数方程组} 的过程即可求 $ax + buy = c$ 的一组解。算法过程如下:

\lstinputlisting[
	style   = cpp,
	caption = {exgcd},
	label   = {exgcd.cpp}
]{contents/exgcd.cpp}

% 考虑通解的情况。若已知解为 $x_1, \phantom{;} y_1$, \phantom{;} 且有 $x_2 = x_1 + k$ 也可以构成解,则有 $a x_2 + a k + b y = c$,即 $a x_2 + b(y + \frac{a}{b} k) = c$。上式中 $k, \phantom{;} \frac{a}{b} k \in \mathbb{Z}$ 才有意义,又有:

% \[
%     \frac{a}{b} = \frac{\frac{a}{\gcd(a, \phantom{;} b)}}{\frac{b}{\gcd(a, \phantom{;} b)}} \quad\text{记为} \frac{a'}{b'}
%     \]

% 则应有 $k = k'b'$,即有通解 $x_k = x_1 + k \frac{b}{\gcd(a, \phantom{;} b)}$,$y_k = x_k - k \frac{a}{\gcd(a, \phantom{;} b)}$。

考虑通解的情况。若已知解为 $x_1, \phantom{;} y_1$ 且 $x_2, \phantom{;} y_2$ 也为一组解,那么两方程相减可得:

\begin{equation}
	\begin{aligned} \notag
		(x_1 - x_2)a + (y_1 - y_2)b                  & = 0                                            \\
		(x_1 - x_2)a                                 & = (y_2 - y_1)b                                 \\
		(x_1 - x_2) \frac{a}{\gcd(a, \phantom{;} b)} & = (y_2 - y_1) \frac{b}{\gcd(a, \phantom{;} b)} \\
	\end{aligned}
\end{equation}

由于 $\frac{a}{\gcd(a, \phantom{;} b)}$ 与 $\frac{b}{\gcd(a, \phantom{;} b)}$ 互素,则有 $\frac{a}{\gcd(a, \phantom{;} b)} \vert (y_2 - y_1)$,$\frac{b}{\gcd(a, \phantom{;} b)} \vert (x_1 - x_2)$,即通解可以写为:

\begin{equation}
	\begin{cases}
		x_k = x_1 + k \frac{b}{\gcd(a, \phantom{;} b)} \\
		y_k = x_k - k \frac{a}{\gcd(a, \phantom{;} b)} \\
	\end{cases}
\end{equation}
\subsection{模的逆}

% 若 $a \cdot x \equiv 1 (\bmod \phantom{;} y)$,则 $a, \phantom{;} x$ 互为模 $y$ 意义下的逆元。变换形式,可得$a \cdot x + b \cdot y = 1$。形如 $a \cdot x + b \cdot y = d$ 的方程被称为裴蜀(Bezout)等式或贝祖等式。$x, \phantom{;} y$已知,则此等式存在整数解$a, \phantom{;} b \iff \gcd(x, \phantom{;} y) \mid d$。 即上式中的 $a, \phantom{;}b$ 有解当且仅当$x, \phantom{;} y$互素。

% 先证上述定理。当 $a \cdot x + b \cdot y$ 时,显然有 $\frac{a \cdot x}{\gcd(x, \phantom{;} y)} + \frac{b \cdot y}{\gcd(x, \phantom{;} y)}$ 恒为整数,可得 $\gcd(x, \phantom{;} y) \mid d$。当 $\gcd(x, \phantom{;} y) \mid d$ 时,因为 $\gcd(\frac{x}{\gcd(x, \phantom{;} y)}, \phantom{;}\frac{y}{\gcd(x, \phantom{;} y)}) = 1$ ,即证 $x, \phantom{;} y$ 互素时, $a \cdot x + b \cdot y = k$恒有解。

若 $a \cdot x \equiv 1 (\bmod \phantom{;} b)$,则 $x$ 为 $a$ 模 $b$ 意义下的逆元。变换形式,可得$a \cdot x + b \cdot y = 1$。由 \ref{裴蜀定理} 节, 当 $\gcd(a, \phantom{;} b) = 1$,也就是 $a, \phantom{;} b$ 互素时逆元一定存在。

% TODO 线性求逆元
\subsection{欧拉函数}

欧拉函数$\varphi(n)$表示小于等于$n$的正整数中与其互质的数字个数。其可表示为:

\[
	\varphi(n)=\begin{cases}
		1, \phantom{;}                                                             & n = 1                                          \\
		n(1- \frac{1}{p_1})(1-\frac{1}{p_2}) \cdots (1-\frac{1}{p_r}), \phantom{;} & n = p_{1}^{k_1} p_{2}^{k_2} \cdots p_{r}^{k_r}
	\end{cases}
\]

$n$一定属于以上某条件,可用归纳法证明:

\begin{itemize}
	\item $n=1, \phantom{;} 2$满足以上条件。
	\item $n \geq 3$,若$n$满足以上条件,若$n+1 = p_{1}^{k_1}p_{2}^{k_2}\cdots p_{r}^{k_r}$则也满足以上条件,若$n+1 \neq p_{1}^{k_1}p_{2}^{k_2}\cdots p_{r}^{k_r}$,那它一定是个合数,可以分解为两个或多个小于$n$的数字乘积。
\end{itemize}

下面考虑$\varphi(n)$:

当$n = p^k$时,不包含$p$的数才能与其互质。而包含$p$的数字有$p, \phantom{;}2\cdot p, \phantom{;}\ldots , \phantom{;}p^{k-1}\cdot p$共$p^{k-1}$个,因此此时$\varphi(n) = p^k - p^{k-1} = p^k (1 - \frac{1}{p} )$。

当$n = p_{1}^{k_1}p_{2}^{k_2}\cdots p_{r}^{k_r}$时,可以通过以下两种方式证明该式。

\subsubsection{使用容斥原理证明}

\begin{sloppypar}
	先考虑$n = p_{1}^{k_1}p_{2}^{k_2}$的情况。在计数$\varphi(n)$的过程中不计数$p_1, \phantom{;} 2p_1, \phantom{;} \ldots, \phantom{;} \lfloor \frac{n}{p_1} \rfloor p_1$,同理对$p_2, \phantom{;} 2p_2, \phantom{;} \ldots, \phantom{;} \lfloor \frac{N}{p_2} \rfloor p_2$也不应该计数,并根据容斥原理(计数时重复计数的部分要扣除)还需要在结果中加上$p_1p_2, \phantom{;} 2p_1p_2, \phantom{;} \cdots, \phantom{;} \lfloor \frac{n}{p_1} \rfloor p_2$的数量。这证明了$\varphi(n)=n(1-\frac{1}{p_1})(1-\frac{1}{p_2})$。
\end{sloppypar}

同理,% $\forall m, \phantom{;} n \in \mathbb{Z\textsuperscript{*}}$,
有$\varphi(mn)=mn(1-\frac{1}{p_{m1}})(1-\frac{1}{p_{m2}}) \cdots (1-\frac{1}{p_{mr}})(1-\frac{1}{p_{n1}})(1-\frac{1}{p_{n2}}) \cdots (1-\frac{1}{p_{nk}}) = \varphi(m)\varphi(n)$,这就证明了$n = p_{1}^{k_1}p_{2}^{k_2} \cdots p_{r}^{k_r}$时,$\varphi(n)=n(1-\frac{1}{p_1})(1-\frac{1}{p_2}) \cdots (1-\frac{1}{p_r})$。

\subsubsection{使用中国剩余定理证明}

先介绍中国剩余定理。它旨在解决形如下式的问题:

\begin{equation}
	\label{中国剩余定理问题方程组}
	\begin{cases}
		x \equiv a_1(\bmod \phantom{;} n_1) \\
		x \equiv a_2(\bmod \phantom{;} n_2) \\
		\cdots                              \\
		x \equiv a_r(\bmod \phantom{;} n_r) \\
	\end{cases}
	(\forall i, \phantom{;} j \leqslant r, \phantom{;} \gcd(n_i, \phantom{;} n_j) = 1)
\end{equation}

这个定理推导基于以下性质:

\begin{enumerate}
	\item \label{CRT性质找数} 若$a \bmod \phantom{;} n_1, \phantom{;}n_2, \phantom{;} \cdots, \phantom{;} n_r = 0$,则$a$为$n_1, \phantom{;}n_2, \phantom{;} \cdots, \phantom{;} n_r$公因子之积的倍数。
	\item \label{CRT性质倍数} 若$a \bmod \phantom{;} n = 1, \phantom{;} k < n$,则$ k \cdot a \bmod \phantom{;} n = k$。
	\item \label{CRT性质拆分} 由$a + k \cdot n \equiv a(\bmod \phantom{;} n)$,可拓展得若$a_1, \phantom{;}a_2, \phantom{;}\cdots, \phantom{;}a_r \bmod \phantom{;} n = 0$,则$a + k_1a_1+k_2a_a+\cdots+k_ra_r \equiv a (\bmod \phantom{;} n)$ % ,进而拓展得$\forall i \in \mathbb{Z \textsuperscript{*}}$ 且 $i \leqslant r, \phantom{;} a_i \bmod \phantom{;} n_i = 1, \phantom{;} a_i \bmod \phantom{;} n_1, \phantom{;}n_2, \phantom{;}\cdots, \phantom{;}n_{i-1}, \phantom{;} n_{i+1}, \phantom{;}\cdots, \phantom{;} n_r = 0$,则$\sum \limits_{i=1}^{r} a_i \bmod \phantom{;} n_1, \phantom{;}n_2, \phantom{;} \cdots , \phantom{;} n_r = 1$。
\end{enumerate}

由性质 \ref{CRT性质拆分} 得我们可以把该问题拆分,形如下式。
% 形如$x_i \equiv a_i(\bmod \phantom{;} n_i) $ 且 $\forall j \neq i, \phantom{;} x_i \equiv 0 (\bmod \phantom{;} n_j)$。

\[
	\begin{cases}
		x_1 \equiv a_1(\bmod \phantom{;} n_1), \phantom{;} x_1 \equiv 0 (\bmod \phantom{;} n_2), \phantom{;} x_1 \equiv 0 (\bmod \phantom{;} n_3), \phantom{;} \cdots, \phantom{;} x_r \equiv 0 (\bmod \phantom{;} n_r) \\
		x_2 \equiv 0(\bmod \phantom{;} n_1), \phantom{;} x_2 \equiv a_2 (\bmod \phantom{;} n_2), \phantom{;} x_2 \equiv 0 (\bmod \phantom{;} n_3), \phantom{;} \cdots, \phantom{;} x_r \equiv 0 (\bmod \phantom{;} n_r) \\
		\vdots                                                                                                                                                                                                          \\
		x_r \equiv 0(\bmod \phantom{;} n_1), \phantom{;} x_r \equiv 0 (\bmod \phantom{;} n_2), \phantom{;} \cdots, \phantom{;} x_r \equiv a_r (\bmod \phantom{;} n_r)                                                   \\
		x = \sum \limits_{i=1}^{r} x_i
	\end{cases}
\]

此时,利用性质 \ref{CRT性质倍数},可以将 $x_i \equiv a_i(\bmod \phantom{;} n_i)$简化为求$x_i' \equiv 1 (\bmod \phantom{;} n_i)$ 且 $x_i = a_i \cdot x_i'$。令$N=n_1 \cdot n_2 \cdots n_r$,则根据性质 \ref{CRT性质找数} 则有$x_i'$满足$x_i = \frac{N}{n_i} \cdot q_i$,且$\frac{N}{n_i} \cdot q_i \equiv 1 (\bmod \phantom{;} n_i)$,即 $q_i$ 为求 $\frac{N}{n_i}$ 模 $n_i$ 的逆,记为$\operatorname{invert}(\frac{N}{n_i}, \phantom{;} n_i)$\upcite{中国剩余定理(CRT)}。此时,方程已经被我们化为如下形式:

\[
	\begin{cases}
		x_1 = a_1\frac{N}{n_1} \cdot \operatorname{invert}(\frac{N}{n_1}, \phantom{;} n_1) \\
		x_2 = a_2\frac{N}{n_2} \cdot \operatorname{invert}(\frac{N}{n_2}, \phantom{;} n_2) \\
		\vdots                                                                             \\
		x_i = a_i\frac{N}{n_i} \cdot \operatorname{invert}(\frac{N}{n_i}, \phantom{;} n_i) \\
		\vdots                                                                             \\
		x_r = a_r\frac{N}{n_r} \cdot \operatorname{invert}(\frac{N}{n_r}, \phantom{;} n_r) \\
		x \equiv \sum \limits_{i=1}^{r} x_i(\bmod \phantom{;} n)
	\end{cases}
\]

这就给出了方程 \ref{中国剩余定理问题方程组} 的解。若$\exists i, \phantom{;} j$,使$\gcd(n_i, \phantom{;} n_j) > 1$,则可通过除其公因子“合并”方程。关于逆元的求法可参考\href{https://blog.csdn.net/xiaoming_p/article/details/79644386}{这个链接}。

下面将用中国剩余定理证明欧拉公式。

设$A = \{ x \vert \gcd(x, \phantom{;} m) = 1, \phantom{;} x \leqslant m \}, \phantom{;} B = \{ y \vert \gcd(y, \phantom{;} n) = 1, \phantom{;} y \leqslant n \}, \phantom{;} C = \{ z \vert \gcd(z, \phantom{;} mn) = 1, \phantom{;} z \leqslant mn \}$,则 $\varphi(m) = \left\lvert A\right\rvert, \phantom{;}\varphi(n) = \left\lvert B\right\rvert, \phantom{;}\varphi(mn) = \left\lvert C\right\rvert$。

$\because \gcd(m, \phantom{;} n) = 1$

$\therefore A \bigcap B = \{ 1 \}$

若$mn \bmod \phantom{;} N = 0$,即$N \in C$,令 $N = k_1m + p = k_2n + q$。

$\because \gcd(N, \phantom{;} mn) = 1$

$\therefore \gcd(N, \phantom{;} m) = \gcd(N, \phantom{;} n) = 1$

$\therefore \gcd(k_1m + p, \phantom{;} m) = gcd(p, \phantom{;} m) = 1, \phantom{;} \gcd(k_2n + q, \phantom{;} n) = gcd(q, \phantom{;} n) = 1, \phantom{;}$

$\therefore \gcd(p, \phantom{;} m) = \gcd(q, \phantom{;} n) = 1$

由中国剩余定理,方程组

\[
	\begin{cases}
		N \equiv p(\bmod \phantom{;} m) \\
		N \equiv q(\bmod \phantom{;} n) \\
	\end{cases}
\]

有通解$N = kmn + pnt_1 + qnt_2$,其中$nt_1 \equiv 1(\bmod \phantom{;} m), \phantom{;} mt_2 \equiv 1 (\bmod \phantom{;} n)$。

因此,对于每一个二元组$(p, \phantom{;} q)$,都有唯一的 $N$ 与之对应 $\Rightarrow A \times B$ 与 $C$ 构成双射。

根据乘法原理,二元组$(p, \phantom{;} q)$的数量为$\varphi(n) \varphi(m)$,而与 $mn$ 互质的数 $N$ 的数量为 $\varphi(mn)$。

$\therefore \varphi(mn) = \varphi(m) \varphi(n)$

$\therefore \forall n \in \mathbb{Z\textsuperscript{*}}, \phantom{;} n = p_1^{k_1}p_2^{k_2} \cdots p_n^{k_n}, \phantom{;} \varphi(n) = p_1^{k_1}(1 - \frac{1}{p_1})p_2^{k_2}(1 - \frac{1}{p_2})\cdots p_n^{k_n}(1 - \frac{1}{p_n})=p_1^{k_1}p_2^{k_2}\cdots p_n^{k_n}(1 - \frac{1}{p_1})(1 - \frac{1}{p_2})\cdots (1 - \frac{1}{p_n})$。

\subsection{欧拉定理}

欧拉定理指的是,如果两个正整数$a$和$n$满足$\gcd(a, \phantom{;} n) = 1$,则有:

\[
	a^{\varphi(n)} \equiv 1 (\bmod \phantom{;} n)
\]

费马小定理是这个定理 $n = p$ 时的特殊形式,即:

\[
	a ^ {p-1} \equiv (\bmod \phantom{;} p)
\]

证明……先等等吧……

\section{RSA数字签名与数字证书}

现在设想这样一个情形:小收公布公钥之后,收到了一条信息要求他给小发转账。于是小收转了50,但第二天,小收却声明他从未发过类似于“vivo50”这样的信息。原来,小攻,作为一名黑客,冒充小发发了一条信息。小收发现了这个公钥体制的问题——没有办法证明信息的发送者。

同一时间,另一头的小攻也在为另一个问题苦恼——他发送的信息明明是“vivo500”,可是他仅仅收到了50!原来,信息在传输的过程中被他的对手小黑(另一名黑客),改成了vivo50!他咬牙切齿,因为小收的消息系统太烂了——根本没办法知道信息发送过程中有没有被篡改。

RSA数字签名与数字证书可证明发信人是其自身,以及消息的完整性。

\subsection{散列函数} \label{散列函数}

验证信息没有被篡改很简单:使用散列函数,对原文进行“摘要”。散列函数应当满足这样的性质:当输入发生微小的变化的时候结果变化很大,并且难以找到有相同输出的输入,这样,当消息发生变化,用相同算法生成的“摘要”也就会发生变化。
% 待补充

\subsection{数字签名与数字证书}

RSA算法原理保证了其有这样的性质:即使使用私钥加密,用公钥也可以解密。这是由第 \ref{RSA数学基础与证明} 章中介绍的数学原理决定的。因此,在上文的例子中,小收也可以制作一份密钥,并将公钥公布。在发送信息的时候,用自己的私钥对摘要进行加密。如果小攻想要修改信息,那么他必须知道小收使用的私钥——不然小发得到的消息和摘要将会不一致。

然而,小攻十分滴聪明狡猾。他用小收的密钥加密了“vivo500”的假消息,然后自己制作了一份公钥发到网上,用私钥对消息的摘要加了密。小收又上了当——这份消息看起来同样是小发的消息,也有“小发的签名”。

这时候需要由数字证书解决这个问题。证书授权(CA, \phantom{;}CertificateAuthority)由相应的CA机构颁发给小收(应经过线下验证),包括了小收的公钥、小收的身份信息、CA的签名。这样,小发就可以找到真正的小收公钥与对应消息。

验证证书的合法性是套娃的过程。小收的公钥与小收的身份信息对应的摘要,应该与CA的解密后签名一致。CA根证书是安装系统内置在系统或浏览器中的,这样如果还想信息造假,需要修改用户的系统或浏览器文件,而这通常是不可能的。

现在如果小攻有能力发送信息、劫持他人发送的信息,且没有能力分解公钥,小攻将无法完成欺骗,除非他可以劫持网络到修改他们发送的信息,或是可线下潜入某位仁兄的房间,修改他们的电脑数据——但如果这样,他直接用他们的账户给自己转账似乎更合理些。

\section{让你的RSA更安全}



\bibliography{contents/rsaref} % 导入bib
\bibliographystyle{plain} % 参考文献排版风格

% \end{sloppypar} % Typeset the paragraphs with \sloppy in effect (see \fussy & \sloppy). Use this to locally adjust line breaking, \phantom{;} to avoid ‘Overfull box’ or ‘Underfull box’ errors.
\end{document}